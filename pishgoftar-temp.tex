\chapter*{پیشگفتار }

پیش‌گفتار، فصلی از پایان‌نامه است که معمولاً شامل بخش یا زیربخش نیست و حدود یک یا دو صفحه است. در آن، مقدمه‌ای به زبان ساده و عاری از فرمول‌بندی ریاضی، از مساله مورد مطالعه در پایان‌نامه ارائه می‌شود. همچنین در پیش‌گفتار است که می‌توان خواننده را با تاریخچه‌ای مختصر از تلاش‌هایی که برای حل مساله مورد مطالعه در پایان‌نامه شده است آشنا نمود و نیز تلاش نمود تا خواننده اهمیت کار انجام شده در پایان‌نامه را دریابد. این قسمت، گرچه در نگاه نخست به ظاهر بسیار ساده می‌نماید، اما در حقیقت یکی از  مهمترین قسمت‌های پایان‌نامه است، زیرا بازتاب دهنده دانسته‌ها و فهم دانشجو از کلیت مساله است. توصیه می‌کنیم که نوشتن این قسمت را به آخر موکول نمایید!
\\
یک نکته مهم: اگر پایان‌نامه بر اساس یک یا چند مقاله نوشته شده است، آنگاه از دانشجو انتظار می‌رود که مراجع اصلی خود را در پیش‌گفتار معرفی نماید.
\\
معمولاً بخش آخر پیش‌گفتار به ارائه یک نمایه کلی از پایان‌نامه اختصاص می‌یابد و نگارنده در پاراگراف‌های مجزا، به معرفی کوتاهی از فصل‌بندی و کار انجام شده در هر فصل پایان‌نامه می‌پردازد.

 

